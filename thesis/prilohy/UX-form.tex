% Lokální makra patří do hlavního souboru, ne sem.
% Tady je mám výjimečně proto, že chci nechat hlavní soubor bez maker,
% která jsou jen pro tento dokument. Uživatelé si pravděpodobně budou
% hlavní soubor kopírovat do svého dokumentu.

\def\ctustyle{{\ssr CTUstyle}}
\def\ttb{\tt\char`\\} % pro tisk kontrolních sekvencí v tabulkách

\label[UXFormAsnwares]
\app Slovní komentáře průzkumu podoby zařízení

Níže přikládám okopírované odpovědi, které nejsou modifikovány mimo vymazání jmen respondentů kvůli ochraně osobních údajů.

\begtt
flákací verze dává možnost vyniknout introvertnějším
ale chytřejším členům týmu, proto vybírám ji
\endtt

\begtt
Hele, v týmu si rozhodně radši praštím.
Kdyby to mělo být třeba na jednotlivce,
nebo nějaký spešl důvod, tak si dovedu představit
ten malej klikač, ale ve většině aplikací bych
upřednostil velký bouchadlo.
Čusákos.
Lobumbus.
\endtt

\begtt
Variantu A bych zvolil pokud by každý hlasoval za sebe,
varianta B mi připadá spíše jako jedno tlačítko
pro celý tým, který stojí kolem.
\endtt

\begtt
V návalu adrenalinu, stejně tak po požití alkoholu stoupá soupeřivost
a určitým způsobem třeba i živelnost či mírná agrese. Proto si myslím,
že odolnější zařízení je pro tento model lepší variantou. 
\endtt

\begtt
Za prvé, nesnáším tyhle hry, ale když si představím, že by
mě to bavilo, tak by bylo víc vzrušující si pořádně
fláknout i za cenu toho, že to může být pomalejší způsob
\endtt

\begtt
Myslím, že při hrách vzniká taková atmosféra,
že odolná buch krabička bude pamparádní.
\endtt

\begtt
Brala bych verzi B - menší šance ztráty a kdo
by chtěl vyměňovat baterky u 30 mini kontrolérů?
Taky na mě možnost B působí víc týmově, nejsi tam sám
za sebe, máš prostě jeden čudlík o který se dělíš se svým týmem.
Obojí se mi zdá v něčem špatné. Malá verze bude jednoduše rozbitná,
ale lepší na ovládání. U druhé zase hrozí nějaké sražení rukou
a možné hádky. Myslím, že by to chtělo kompromis mezi oběma.
Myslím asi malé zařízení, které bude zároveň odolnější
z tvrdsiho plastu a jednoduššího tvaru. 
\endtt

\begtt
Odolnější mi připadá jako v týmu větší zábava, raději
bych teda byl pro tuto variantu s tím, když budou v týmu
hezky slečny tak se můžeme mačkat. 
Pokud to má být zařízení je do hospody a ne na tajné hlasování,
tak určitě odolná krabice na stole uprostřed. Už ten střet rukou
v rámci týmu pak může být zajímavý
\endtt

\begtt
Kromě toho, že bych se musela na flaknuti zvednout, tak by
jeste docházelo k trapasum typu "omlouvám se, vy první"
"ne, vy první, máte přednost" apod. 
\endtt

\begtt
Hodně hustý! Za mě tlačítko uprostřed stolu ideál,
můžeš do toho mlátit jak chceš a nestane se že se třeba omylem
uklikneš na nějaký malý blbosti v ruce, že ti to
upadne nebo Bůh ví co
\endtt

\begtt
Pokud je to pro tým, přijde mi efektivnější zařízení,
ke kterému mají všichni stejnou možnost přístupu + pocitově
to vypadá, že to bude fungovat rychleji než klikání.
\endtt

\begtt
"Ahoj, raději budu mít svoje tlačítko, než kdybych se měl
dělit popřípadě se nějakým způsobem ""prát"" s ostatními,
kdo to zvládne první. Snad to chápu dobře. :D 
\endtt

\begtt
Pro mě to prostě byla první možnost, která mně osobně přišla
pohodlnější… jinak ke svému výběru žádný větší důvod nemám.
\endtt

\begtt
Druhá varianta je asi zajímavější a i na zuřivost po
odpovědi efektivnější, u té první není moc vysvětleno,
jestli by to měl každý hráč v týmu nebo by bylo pouze
jedno klikátko :) z dlouhodobějšího hlediska výdrže
jsem pro kliknutí :D
\endtt

\begtt
Maly do ruky je praktictejsi
\endtt

\begtt
Rozhodně odolnější verze…pak se někdo zapomene,
flákne a ejhle, je po klikací verzi
\endtt

\begtt
Podle mě záleží na cilovce, pro menší děti určitě kravička
a pro dospělé asi ten clicker
\endtt

\begtt
Kliknutí se zdá o dost víc praktické, avšak dle mého názoru
fláknutí k těmto typům her prostě patří a je to větší zábava.
Navíc u klikru by mi vadilo, že nemám volnou ruku a ještě bych
si ho někam položila a ztratila :D
Flákací, to může být uprostřed stolu a lépe se s tím pracuje
v týmu, než neustále ztrácet čas zjišťováním, u koho zůstal ovladač 
\endtt

\begtt
Myslím, že nejdokonalejší verze by byla, kdyby měl každý své
odolné tlačítko. Ze začátku hry by jistě stačilo jen tak lehce klikat,
ale postupem času, kdy se do soutěže pořádně vžijete, je potřeba
si řádně fláknout, co nejrychleji a nejlépe tak, aby ostatní viděli,
že jste byl první. Proto by se hodilo, aby měl každý své odolné
tlačítko a nemusel se “strachovat”, že se nestihne včas
natáhnout do prostřed stolu nebo že mu bude někdo kolem překážet.
Ale kdyby taková možnost nebyla, tak aspoň to jedno odolné tlačítko
uprostřed, aby se člověk nebál, že to malé tlačítko v zápalu hry
rozbije :) V zapálu boje bych si vybral druhou možnost,
protože ta déle vydrží.
\endtt

\begtt
Rozhodně fláknu! To musí být nejzábavnější část toho procesu,
jinak nehraju. :D Zdravím z Brna a krásný den:)
\endtt

\begtt
Proste jako Cink, to je vždycky adrenalin a hlavně by jsi do toho
mohl vstupovat s nějakejma pravidlama - např. všichni musí mít ruce
za zády, dokud se neřekne teď.. atd. což by u toho clickeru nešlo.
Takže ja jsem proc řco jako cink, idealně i se sound effectem
Ať se daří! 
\endtt

\begtt
Emoce přidávají na síle. Čím robustnější, tím lepší.
\endtt

\begtt
Pokud by měl každý své zařízení pak jsem pro kliknutí
Pokud by bylo jedno, tak fláknutí.
\endtt

\begtt
1) Volím flákanec, protože když jsem ve hře, tak mám
extra adrenalin a určitě je uspokojující flaknout do tlačítka
celou dlaní než mačkat tlačítko.
2) Přijde mi to lepší, protože je jedno, zda sedíš nebo stojíš,
prostě je tato varianta přístupná krásně všem a může hádat víc
lidí najednou.
3) Ve většině her musíš mít ruce za zádama a tak je tato
varianta lepší. 
4) Je to rozhodně lepší varianta i s přihlédnutím k věku lidí.
Malé děti mají menší ruku a proto můžou mít problém se
stisknutím tlačítka, které bude ovládat pouze palec.
5) S tímto tlačítkem se bude pracovat nejen malým dětem,
ale také seniorům (kteří mají ruce, které se klepají a ochablý
palec - mohli by tlačítko zmáčknout nechtěně) a postiženým
(kteří nemají takovou citlivost v prstech a spíš ovládají
hrubou motoriku)
\endtt

\begtt
Záleží na co to bude použité - pokud pro hospodský kvíz,
tak máš víc lidí v týmu a kdo z nich bude první vědět tak
do toho flákne. A je třeba, aby to bylo ideálně velké
a dobře viditelné, protože nějakýho malýho švába by taky
mohli ty lidi v hospodě I pár sekund hledat. 
Když to bude něco individuálního, třeba i běhacího, tak
tam se hodí, že je to malé, můžu to mít permanentně
v ruce a kliknout kdy budu potřebovat.
Takže obě dvě verze jsou možné a každá řeší něco jinýho
a to si musíš říct ty, co preferuješ, jaký přístup. V bakalářce
bych ale zmínil oba a uvedl, že u tohodle mám ale prototyp.
A kdybys potřeboval 3D tiskárnu na prototypování, tak neváhej
napsat, moje je ti kdykoliv k dispozici! 
Pokud chceš dost dobrý informace, tak si napiš Zdeňku Mikovcovi
email: xmikovec@fel.cvut.cz je to garant celýho HCI oboru
a je totálně povídavej, takže myslím, že kdyby sis mu napsal,
tak se s ním můžeš v pohodě sejít a dá ti taky know-how.
\endtt

\begtt
Pokud půjde o rychlost, je lepší mít něco v ruce. Vzhledem
k mé práci s dětmi vím, že u "flákacího" zařízení si hodně
ublíží (i fyzicky i slovně), tak bych spíše volila
pro každého jedno. 
\endtt

\begtt
odolnost je v hospodě důležitá
\endtt

\begtt
Při hraní her bych rozhodně raději flákla, protože při
hospodském kvízu, ale i jiných stolních hrách, které se
hrají povětšinou s přáteli je žádoucí dát průchod svým emocím,
což mi prezentér absolutně neumožní. Zároveň by musel mít
v ruce každý svůj (více práce i nákladů), protože do prezentétu
nelze mlátit (skákal by po stole a tlačítko by se možná ani
nezmáčklo). Prezentér je tedy nutné vzít do ruky a pak až
mačkat (za éře více času), což je dle mého názoru mnohem
méně atraktivni možnost (nelze se při ní vyblbnout).
Doporučila bych ji třeba na nějaké konference či jednání,
kde se chce člověk přihlásit o slovo. 
\endtt

\begtt
Zároveň mám poznámku k “flákacímu” zařízení. Udělala bych určitě
větší tlačítko a vypouklé, aby hezky sedělo do dlaně. Ale chápu,
že může být obtížnější jej sehnat. Možná by se dalo odmontovat
z nějaké hračky
\endtt

\begtt
Vzhledem k adrenalinu, napětí a vzrušení, který panuje v týmech
bych zvolil robustnější krabičku.
PS: radši si fláknu"
\endtt

\begtt
Myslím, že by bylo lepší odolnější zařízení, možná trochu
nižší než na obrázku, aby na něj lidi lépe dosáhli ale může
to dodat tu "pravou" soutěživost, a zároveň dlouho vydrží
\endtt

\begtt
Jelikož se bude hlásit o slovo celý tým, který má více lidí,
tak by asi chtělo, aby bylo jedno velké tlačítko uprostřed stolu.
\endtt

\begtt
Idk jestli to dobře chápu, ale to B vypadá víc týmově
\endtt

\begtt
Když klikneš jen sám u sebe, je to méně kolektivní. Když
se lidi perou o tlačítko, které je uprostřed, přijde mi
to jako větší zábava. Ale samozřejmě záleží dle druhu
a cíle hry.
\endtt

\begtt
Většinou jsou lidé do takových her tak zapálení, že vy
se mohlo stát, že by klikací verzi rozbili.
\endtt

\begtt
Párkrát jsem na hospodském kvízu byla a byla to taková
přetlačovaná oslava kdo křiknul první dobrou odpověď byl
vítěz takže pro mě kliknutí by bylo dobré na té středové
krabičce by byla asi rvačka
\endtt

\begtt
B funguje pro všechny členy týmu. Pro určitě skupiny
lidí nebezpečné, že se lidi v týmu platí navzájem.
Záleží, jaká je cílovka...
A by musel mačkat jeden, takže když někdo z týmu ví,
musí nejdřív předat pokyn k přihlášení... Pomalejší,
kultivovanější.
\endtt

\begtt
Možnost B. Větší věc, menší možnost ztráty. Všichni
uvidí, jaký člen z týmu se přihlásil. Navíc by se jich
nemuselo vyrábět tolik, sice je to větší zařízení, ale
tu malou by musel mít každý člen týmu. Velká krabice
stačí jedna doprostřed stolu. Bylo by možné, že je jen
jedno tlačítko do týmu. Mohl by se tak hlásit každý
a ne jen ten, kdo by ji držel. Navíc při vypjatější
atmosféře to bouchnutí k tomu prostě tak nějak patří.
\endtt

\begtt
Pokud je účel pobavení a nezáleží tolik na presnosti.
Tak B, záleží na přesném použití za mě
\endtt

\begtt
Cítím, že tvůj favorit je flákadlo, přesto bych
uživatelsky volil klik. Uznávám, že flák je divácky
a adrenalinově zajímavější. Kdybych se ale měl rychle
přihlásit, už na tom chci držet palec. Škoda žes nepřidal
aspoň další otázky pohlaví a věk. Mohlo by se to
vyprofilovat a pro různý druhy příležitostí mít víc
variant hlasovacích zařízení. Předpokládám, že chlapi
budou raději flákat a s věkem bude stoupat klikavost. 
\endtt