% Lokální makra patří do hlavního souboru, ne sem.
% Tady je mám výjimečně proto, že chci nechat hlavní soubor bez maker,
% která jsou jen pro tento dokument. Uživatelé si pravděpodobně budou
% hlavní soubor kopírovat do svého dokumentu.

\def\ctustyle{{\ssr CTUstyle}}
\def\ttb{\tt\char`\\} % pro tisk kontrolních sekvencí v tabulkách

\label[infrasmeasuredetails]
\app Detailní výsledky měření síťové infrastruktury

\sec Scénář A

\medskip
\clabel[ScenarioA-img]{Vizualizace parametrů pro měření scénářů A}
\picw=14cm \cinspic img/ScenarioA.png
\caption/f Vizualizace parametrů pro měření scénářů A.
\medskip

Ve scénářích typu A se snažím otestovat to, jak změna parametru typu odesílání, tj. přepínání mezi {\em brodacastem} a {\em unicastem}, ovlivní {\em round-time trip} v závislosti na velikosti zprávy. Měření se odehrává v prostředí bytu v činžovním domě, kde dochází k rušení několika okolními Wifi sítěmi.

\midinsert \clabel[ScenariosA1-tab]{Přehled parametrů pro A1x scénáře}
\ctable{lrrrrr}{
\hfil {\sbf SCÉNÁŘE} & {\sbf A11} & {\sbf A12} & {\sbf A13} & {\sbf A14} & {\sbf A15}\crl
{\sbf prostředí} & byt & byt & byt & byt & byt\cr
{\sbf překážka} & vzduch & vzduch & vzduch & vzduch & vzduch\cr
{\sbf vzdálenost} & 50 cm & 50 cm & 50 cm & 50 cm & 50 cm\cr
{\sbf velikost} & 1 B & 10 B & 50 B & 120 B & 250 B\cr
{\sbf počet zpráv} & 10 000 & 10 000 & 10 000 & 1 000 & 5 000\cr
{\sbf typ vysílání} & broadcast & broadcast & broadcast & broadcast & broadcast\cr
}
\caption/t Přehled parametrů pro A1x scénáře.
\endinsert

\midinsert \clabel[ScenariosA2-tab]{Přehled parametrů pro A2x scénáře}
\ctable{lrrrrr}{
\hfil {\sbf SCÉNÁŘE} & {\sbf A21} & {\sbf A22} & {\sbf A23} & {\sbf A24} & {\sbf A25}\crl
{\sbf prostředí} & byt & byt & byt & byt & byt\cr
{\sbf překážka} & vzduch & vzduch & vzduch & vzduch & vzduch\cr
{\sbf vzdálenost} & 50 cm & 50 cm & 50 cm & 50 cm & 50 cm\cr
{\sbf velikost} & 1 B & 10 B & 50 B & 120 B & 250 B\cr
{\sbf počet zpráv} & 10 000 & 10 000 & 10 000 & 10 000 & 10 000\cr
{\sbf typ vysílání} & unicast & unicast & unicast & unicast & unicast\cr
}
\caption/t Přehled parametrů pro A2x scénáře.
\endinsert

Výsledky měření těchto scénářů si je možné prohlédnout v grafech \ref[ScenarioA1-graph] a \ref[ScenarioA2-graph]. Odesílání se chová dle očekávání. Zprávy větší velikosti trvají déle než zprávy té menší. Zajímavé je srovnání {\em broadcastu} a {\em unicastu}. {\em Broadcast} je nepatrně pomalejší.

Také je zajímavé si povšimnout nakumulovaných odpovědí v jeden čas. Tuto skutečnost si vysvětluji implementací {\em broadcastu} v protokolu ESP-NOW.

\medskip
\clabel[ScenarioA1-graph]{Graf výsledků měření scénářů A1}
\picw=16cm \cinspic img/ScenarioA1-graph.png
\caption/f Graf výsledků měření scénářů A1.
\medskip

\medskip
\clabel[ScenarioA2-graph]{Graf výsledků měření scénářů A2}
\picw=16cm \cinspic img/ScenarioA2-graph.png
\caption/f Graf výsledků měření scénářů A2.
\medskip

\sec Scénář C

\medskip
\clabel[ScenarioC-img]{Vizualizace parametrů pro měření scénářů C}
\picw=14cm \cinspic img/ScenarioC.png
\caption/f Vizualizace parametrů pro měření scénářů C.
\medskip

Sadou scénářů Cx se snažím pozorovat vlastnosti v signálově čistém prostředí\fnote{Nejedná se o laboratorně čisté prostředí. Měření bylo prováděno v prostředí lesa, který je od nejbližší obce vzdálen asi 5 km a v okolí mého bydliště nejvíce čisté od 2,4 GHz rušení.} v závislosti na vzdálenosti a typu vysílání.

\midinsert \clabel[ScenariosC-tab]{Přehled parametrů pro Cx scénáře}
\ctable{lrrrrrr}{
\hfil {\sbf SCÉNÁŘE} & {\sbf C1} & {\sbf C2} & {\sbf C3} & {\sbf C4} & {\sbf C5} & {\sbf C6}\crl
{\sbf prostředí} & les & les  & les  & les  & les  & les\cr
{\sbf překážka} & vzduch & vzduch & vzduch & vzduch & vzduch\cr
{\sbf vzdálenost} & 0,5 m & 25 m & 50 m & 100 m & 100 m & 50 m\cr
{\sbf velikost} & 125 B & 125 B & 125 B & 125 B & 125 B & 125 B\cr
{\sbf počet zpráv} & 5 000 & 5 000 & 5 000 & 5 000 & 5 000 & 5 000\cr
{\sbf počet chybných zpráv} & 0 & 3 & 15 & 35 & 15 & 6\cr
{\sbf typ vysílání} & unicast & unicast & unicast & unicast & broadcast & broadcast\cr
}
\caption/t Přehled parametrů pro Cx scénáře.
\endinsert

Výsledky měření těchto scénářů si je možné prohlédnout v grafu \ref[ScenarioC-graph]. Zde se opět protokol chová dle očekávání. Nedochází k takovému zpoždění jako například při měření scénářů typu A. Také si je možné povšimnou toho, že se jednou za čas nějaká zpráva opozdí.

Při tomto měření jsem zaznamenával také chybovost počet chybných zpráv.\fnote{Chybnou zprávou se myslí taková zpráva, která nedorazí do specifikovaného deadlinu, tedy mimo graf.} Jejich četnost si ji možné prohlédnout v tabulce \ref[ScenariosC-tab].

Během tohoto měření jsem zjistil, že je důležité, aby na větší vzdálenosti\fnote{25 m a více} nestála signálu v cestě žádná překážka.

Zajímavé je také srovnání scénáře A a C. Můžeme pozorovat, že {\sbf vliv vzdálenosti ovlivňuje především ztrátovost paketů. Naopak velikost ovlivňuje rychlost přenosu.}


\medskip
\clabel[ScenarioC-graph]{Graf výsledků měření scénářů C}
\picw=16cm \cinspic img/ScenarioC-graph.png
\caption/f Graf výsledků měření scénářů C.
\medskip

\sec Scénář D

\medskip
\clabel[ScenarioD-img]{Vizualizace parametrů pro měření scénářů D}
\picw=14cm \cinspic img/ScenarioD.png
\caption/f Vizualizace parametrů pro měření scénářů D.
\medskip

Scénář D byl oproti ostatním měřením odlišný v tom, že jsem se nejprve snažil stanovit hranici, kdy je zařízení ještě schopno přijímat zprávy a kdy už ne. Experimentálně jsem dospěl k hranici 580 m. Následně jsem odeslal 1000 zpráv s cílem zjistit, jak je veliká ztrátovost. Měření bylo realizováno na poli, přes které může procházet signál na 2,4 GHz.

\midinsert \clabel[ScenariosD-tab]{Přehled parametrů pro D scénář}
\ctable{lr}{
\hfil {\sbf SCÉNÁŘE} & {\sbf D1}\crl
{\sbf prostředí} & pole (s 2,4 GHz)\cr
{\sbf překážka} & vzduch\cr
{\sbf vzdálenost} & 577 m\cr
{\sbf velikost} & 125 B\cr
{\sbf počet zpráv} & 1 000\cr
{\sbf počet chybných zpráv} & 50\cr
{\sbf typ vysílání} & unicast\cr
}
\caption/t Přehled parametrů pro D scénář.
\endinsert

Při měření jsem zjistil, že při odesílání na velikou vzdálenost je třeba dbát na orientaci čipu. Pokud nebylo zařízení správně natočeno, nešlo odeslat žádné zprávy.

Výsledek měření je vizualizován grafem \ref[ScenarioD-graph]. Při tomto scénáři bylo ovšem mnohem zajímavější pozorovat četnost úplné ztráty dat.

Při odeslání 1000 zpráv, se ztratilo 50. Můžeme tedy jednoduchým výpočtem zjistit, jaká je procentuální ztrátovost na dlouhé vzdálenosti.

 $$
 loss = {error\over sent}
 $$

V našem případě se při odeslání 1000 zpráv objevilo 50 chyb. Chybovost je tedy 5~\%.\fnote{${50\over 1000} = 0,05$}.

\medskip
\clabel[ScenarioD-graph]{Graf výsledků měření scénáře D}
\picw=16cm \cinspic img/ScenarioD-graph.png
\caption/f Graf výsledků měření scénáře D.
\medskip