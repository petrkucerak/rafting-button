
% !TEX root = ctustyle-doc.tex

% The documentation of the usage of CTUstyle -- the template for
% typessetting thesis by plain\TeX at CTU in Prague
% ---------------------------------------------------------------------
% Petr Olsak  Jan. 2013

% You can copy this file to your own file and do some changes.
% Then you can run:  optex your-file

\input ctustyle3  % The template (in version 3, for OpTeX) is included here.

\worktype [O/CZ] % Type: B = bachelor, M = master, D = Ph.D., O = other
                 % / the language: CZ = Czech, SK = Slovak, EN = English

\faculty    {F3}  % Type your faculty F1, F2, F3, etc. or MUVS
            % use main language of your document here:
\department {Katedra měření}
\title      {Využití Wifi v IoT aplikacích\nl a testování specifických vlastností}
\subtitle   {}
            % \subtitle is optional
\author     {Petr Kučera}
\authorinfo {}
\date       {Září 2022, leden 2023}
\supervisor {doc. Ing. Jiří Novák, Ph.D.}  % One or more supervisors
\studyinfo  {Otevřená informatika}  % Study programme etc.
\workname   {Samostatný projekt} % Used only if \worktype [O/*] (Other)
            % optional more information about the document:
\workinfo   {\url{https://github.com/petrkucerak/rafting-button}}
            % Title / Subtitle in minor language:
\titleEN    {Utilization of Wifi in IoT applications and testing specific properties}
\subtitleEN {}
            % If minor language is other than English
            % use \titleCZ, \subtitleCZ or \titleSK, \subtitleSK instead it.
\pagetwo    {}  % The text printed on the page 2 at the bottom.

\abstractEN {
   The goal of the individual project is to familiarize oneself with the possibilities of Wifi technology for IoT applications. Then, to specify critical parameters for application in a subsequent bachelor thesis, to select the appropriate device and test the required properties.
}
\abstractCZ {
   Cílem samostatného projektu je se seznámit s možnostmi Wifi technologie pro IoT aplikace. Následně vyspecifikovat kritické parametry pro aplikaci v navazující bakalářské práci, zvolit správné zařízení a otestovat požadované vlastnosti.
}           % If your language is Slovak use \abstractSK instead \abstractCZ

\keywordsEN {%
   IoT; Wifi; embedded; microcontroller; individual project; wireless communication; distributed system.
}
\keywordsCZ {%
   IoT; Wifi; embedded; mikrokontroler; samostatný projekt; bezdrátová komunikace; distribuovaný systém.
}
\thanks {           % Use main language here
    Chtěl bych poděkovat svému vedoucímu práce doc. Ing. Jiřímu Novákovi, Ph.D. za trpělivost při hledání důvodů poškození modulů a pomoc se směřováním projektu. Svému bratrovi Jakubovi za pomoci při měření vlastností protokolu ESP-NOW v prostředí divoké přírody a parádní fotografie. Mým kamarádům, spolubydlícím a tátovi za dlouhé noční diskuze, které mne vždy v projektu nasměrovali dalším směrem.
}
\declaration {      % Use main language here
   Prohlašuji, že jsem předloženou práci vypracoval
   samostatně a že jsem uvedl veškeré použité informační zdroje v~souladu
   s~Metodickým pokynem o~dodržování etických principů při přípravě
   vysokoškolských závěrečných prací.

   Ve Spojil dne 20. 1. 2023 % !!! Attention, you have to change this item.
   \signature % makes dots
}

%%%%% <--   % The place for your own macros is here.

%\draft     % Uncomment this if the version of your document is working only.
%\linespacing=1.7  % uncomment this if you need more spaces between lines
                   % Warning: this works only when \draft is activated!
%\savetoner        % Turns off the lightBlue backround of tables and
                   % verbatims, only for \draft version.
%\blackwhite       % Use this if you need really Black+White thesis.
%\onesideprinting  % Use this if you really don't use duplex printing. 

\makefront  % Mandatory command. Makes title page, acknowledgment, contents etc.

\input chaps/introduce
\input chaps/reserse
\input chaps/pozadavky
\input chaps/realizace
\input chaps/mereni
\input chaps/zaver
\input prilohy

\bye
