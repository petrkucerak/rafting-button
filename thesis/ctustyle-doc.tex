% The documentation of the usage of CTUstyle -- the template for
% typessetting thesis by plain\TeX at CTU in Prague
% ---------------------------------------------------------------------
% Petr Olsak  Jan. 2013

% You can copy this file to your own file and do some changes.
% Then you can run:  optex your-file

\input ctustyle3  % The template (in version 3, for OpTeX) is included here.

\worktype [B/CZ] % Type: B = bachelor, M = master, D = Ph.D., O = other
                 % / the language: CZ = Czech, SK = Slovak, EN = English

\faculty    {F3}  % Type your faculty F1, F2, F3, etc. or MUVS
            % use main language of your document here:
\department {Katedra měření}
\title      {Distribuovaný systém IoT zařízení řešící problém konsenzu}
\subtitle   {}
            % \subtitle is optional
\author     {Petr Kučera}
\authorinfo {Internet věcí}
\date       {Květen 2022, leden 2023}
\supervisor {doc. Ing. Jiří Novák, Ph.D.}  % One or more supervisors
\studyinfo  {Otevřená informatika}  % Study programme etc.
\workname   {} % Used only if \worktype [O/*] (Other)
            % optional more information about the document:
\workinfo   {\url{https://github.com/petrkucerak/rafting-button}}
            % Title / Subtitle in minor language:
\titleEN    {A distributed system of IoT devices solving the consensus problem}
\subtitleEN {}
            % If minor language is other than English
            % use \titleCZ, \subtitleCZ or \titleSK, \subtitleSK instead it.
\pagetwo    {}  % The text printed on the page 2 at the bottom.

\abstractEN {
   This work deals with the issue of consensus in distributed IoT networks. The goal is to create a concept for a device that will be able to arrange individual events according to order in such a way that their causality is preserved.
}
\abstractCZ {
   Tato práce se zabývá problematikou konsenzu v distribuovaných IoT sítích. Konkrétně si klade za cíl vytvořit koncept zařízení, které se bude schopné uspořádat jednotlivé události podle pořadí tak, aby byla zachována jejich kauzalita.
}           % If your language is Slovak use \abstractSK instead \abstractCZ

\keywordsEN {%
   distributed system, consensus, IoT, ESP-NOW, ESP32, time synchronization
}
\keywordsCZ {%
   distribuovaný systém, konsezus, IoT, ESP-NOW, ESP32, synchronizace času
}
\thanks {           % Use main language here
    Chtěl bych poděkovat vedoucímu mé bakalářské práce doc. Ing. Jiřímu Novákovi, Ph.D. za trpělivost a veškerý čas, který do mne investoval. Svému bratrovi Jakubovi za pomoci při měření vlastností protokolu síťové infrastruktury v prostředí divoké přírody a parádní fotografie. Rád bych také poděkoval své rodině, kamarádům a spolubydlícím za podporu během mého studia.
}
\declaration {      % Use main language here
   Prohlašuji, že jsem předloženou práci vypracoval
   samostatně a že jsem uvedl veškeré použité informační zdroje v~souladu
   s~Metodickým pokynem o~dodržování etických principů při přípravě
   vysokoškolských závěrečných prací.

   V Praze dne 26. 5. 2023 % !!! TODO: Attention, you have to change this item.
   \signature % makes dots
}

%%%%% <--   % The place for your own macros is here.

%\draft     % Uncomment this if the version of your document is working only.
%\linespacing=1.7  % uncomment this if you need more spaces between lines
                   % Warning: this works only when \draft is activated!
%\savetoner        % Turns off the lightBlue backround of tables and
                   % verbatims, only for \draft version.
%\blackwhite       % Use this if you need really Black+White thesis.
%\onesideprinting  % Use this if you really don't use duplex printing. 

\specification {%
   \vbox to0pt{\vskip-25mm\centerline{\inspic prilohy/zadani.pdf }\vss}
}
\makefront  % Mandatory command. Makes title page, acknowledgment, contents etc.

\input chaps/01_introduction
\input chaps/02_reserse
\input chaps/03_pozadavky
\input chaps/04_infrastruktura
\input chaps/05_algoritmus
\input chaps/06_obal
\input chaps/07_zaver

\input prilohy

\bye
