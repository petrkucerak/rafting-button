% Lokální makra patří do hlavního souboru, ne sem.
% Tady je mám výjimečně proto, že chci nechat hlavní soubor bez maker,
% která jsou jen pro tento dokument. Uživatelé si pravděpodobně budou
% hlavní soubor kopírovat do svého dokumentu.

\def\ctustyle{{\ssr CTUstyle}}
\def\ttb{\tt\char`\\} % pro tisk kontrolních sekvencí v tabulkách

\chap Měření vlastností síťové infrastruktury

Kapitola popisuje prostředí, v kterých byla měřena data a způsob jejich měření. Vykresluje jednotlivé scénáře a interpretuje naměřená data v souvislostech.

\label[UvodKMereni]
\sec Úvod k měření

Abych byl schopen zajistit plynulé fungování distribuované sítě, je třeba otestovat vlastnosti protokolu ESP-NOW. Proto jsem v závěru samostatného projektu vytvořil jednoduchý program pro testování. Funguje tak, že:

1. zaregistruje USB callback a spustí sériovou komunikaci,

2. inicializuje struktury nutné pro správné fungování protokolu ESP-NOW\fnote{Jako je například `ESP_NOW_HANDLER`} a zaregistruje callback pro příchozí a odchozí komunikaci.

3. V případě veze MCU s displejem,\fnote{ESP32-S2-pico-LCD} inicializuje displej a vypíše na něj vlastni MAC adresu.

4. Ukončí inicializační metodu. V případě MCU bez displaje\fnote{ESP32-S2-pico} ohlásí konec konfigurace bliknutím červené LED diody, která je vestavěné na desce.

5. Spustí měření, které má za úkol simulovat reálný provoz. Zde je možné konfigurovat základní parametry\fnote{Více je specifikováno v sekci \ref[ScenareMereni].}. Program měří {\em round-time trip}. Respektive dobu, poslání informace ze zařízení A, jejího uložení do patřičné struktury v zařízení B, odeslání zpět do zařízení A a uložení do patřičné struktury v zařízení A. Pokud se informace nevrátí ze zařízení B do určitého deadlinu, zařízení A registruje zprávu jako chybovou a odesílá rovnou další. V případě použití MCU s displejem, je proces měření zobrazován ve formě {\em progress baru}.



6. Po ukončení měření, se data vypíšou do konzole pro následné zpracování.


\label[ScenareMereni]
\sec Scénáře měření

Aby bylo možné otestovat různé parametry v různých prostředích, rozhrnul jsem měření na několik scénářů.

Při měření byly sledovány následující parametry:

\begitems
* prostředí, v němž bylo měření prováděno,
* překážka mezi zařízeními,
* vzdálenost mezi zařízeními,
* velikost zprávy,\fnote{Někdy označováno také jako {\em payload}.} \fnote{Maximální velikost zprávy je 250 bajtů.}
* počet odeslaných zpráv,
* typ odesílání ({\em broadcast}/{\em unicast}),
* počet chybných zpráv.\fnote{Měření tohoto parametru bylo přidáno od scénáře A, kde ovšem k žádné ztrátě nedocházelo.}
\enditems

\secc Scénář A

\medskip
\clabel[ScenarioA-img]{Vizualizace parametrů pro měření scénářů A}
\picw=14cm \cinspic img/ScenarioA.png
\caption/f Vizualizace parametrů pro měření scénářů A.
\medskip

V scénářích typu A se snažím otestovat to, jak změna parametru typu odesílání, tj. přepínání mezi {\em brodacastem} a {\em unicastem}, ovlivní {\em round-time trip} v závislosti na velikosti zprávy. Měření se odehrává v prostředí bytu v činžovním domě, kde dochází k rušení několika okolními Wifi sítěmi.

\midinsert \clabel[ScenariosA1-tab]{Přehled parametrů pro A1x scénáře}
\ctable{lrrrrr}{
\hfil {\sbf SCÉNÁŘE} & {\sbf A11} & {\sbf A12} & {\sbf A13} & {\sbf A14} & {\sbf A15}\crl
{\sbf prostředí} & byt & byt & byt & byt & byt\cr
{\sbf překážka} & vzduch & vzduch & vzduch & vzduch & vzduch\cr
{\sbf vzdálenost} & 50 cm & 50 cm & 50 cm & 50 cm & 50 cm\cr
{\sbf velikost} & 1 B & 10 B & 50 B & 120 B & 250 B\cr
{\sbf počet zpráv} & 10 000 & 10 000 & 10 000 & 1 000 & 5 000\cr
{\sbf typ vysílání} & broadcast & broadcast & broadcast & broadcast & broadcast\cr
}
\caption/t Přehled parametrů pro A1x scénáře.
\endinsert

\midinsert \clabel[ScenariosA2-tab]{Přehled parametrů pro A2x scénáře}
\ctable{lrrrrr}{
\hfil {\sbf SCÉNÁŘE} & {\sbf A21} & {\sbf A22} & {\sbf A23} & {\sbf A24} & {\sbf A25}\crl
{\sbf prostředí} & byt & byt & byt & byt & byt\cr
{\sbf překážka} & vzduch & vzduch & vzduch & vzduch & vzduch\cr
{\sbf vzdálenost} & 50 cm & 50 cm & 50 cm & 50 cm & 50 cm\cr
{\sbf velikost} & 1 B & 10 B & 50 B & 120 B & 250 B\cr
{\sbf počet zpráv} & 10 000 & 10 000 & 10 000 & 10 000 & 10 000\cr
{\sbf typ vysílání} & unicast & unicast & unicast & unicast & unicast\cr
}
\caption/t Přehled parametrů pro A2x scénáře.
\endinsert

Výsledky měření těchto scénářů si je možné prohlédnout v grafech \ref[ScenarioA1-graph] a \ref[ScenarioA2-graph]. Odesílání se chová dle očekávání. Zprávy větší velikosti trvají déle než zprávy té menší. Zajímavé je srovnání {\em broadcastu} a {\em unicastu}. {\em Broadcast} je nepatrně pomalejší.

Také je zajímavé si povšimnout nakumulovaných odpovědí v jeden čas. Tuto skutečnost si vysvětluji implementací {\em broadcastu} v protokolu ESP-NOW.

%\secc Scénář B

%\medskip
%\clabel[ScenarioB-img]{Vizualizace parametrů pro měření scénářů B}
%\picw=14cm \cinspic img/ScenarioB.png
%\caption/f Vizualizace parametrů pro měření scénářů B.
%\medskip

\secc Scénář C

\medskip
\clabel[ScenarioC-img]{Vizualizace parametrů pro měření scénářů C}
\picw=14cm \cinspic img/ScenarioC.png
\caption/f Vizualizace parametrů pro měření scénářů C.
\medskip

Sadou scénářů Cx se snažím pozorovat vlastnosti v signálově čistém prostředí\fnote{Nejedná se o laboratorně čisté prostředí. Měření bylo prováděno v prostředí lesa, který je od nejbližší obce vzdálen asi 5 km a v okolí mého bydliště nejvíce čisté od 2,4 GHz rušení.} v závislosti na vzdálenosti a typu vysílání.

\midinsert \clabel[ScenariosC-tab]{Přehled parametrů pro Cx scénáře}
\ctable{lrrrrrr}{
\hfil {\sbf SCÉNÁŘE} & {\sbf C1} & {\sbf C2} & {\sbf C3} & {\sbf C4} & {\sbf C5} & {\sbf C6}\crl
{\sbf prostředí} & les & les  & les  & les  & les  & les\cr
{\sbf překážka} & vzduch & vzduch & vzduch & vzduch & vzduch\cr
{\sbf vzdálenost} & 0,5 m & 25 m & 50 m & 100 m & 100 m & 50 m\cr
{\sbf velikost} & 125 B & 125 B & 125 B & 125 B & 125 B & 125 B\cr
{\sbf počet zpráv} & 5 000 & 5 000 & 5 000 & 5 000 & 5 000 & 5 000\cr
{\sbf počet chybných zpráv} & 0 & 3 & 15 & 35 & 15 & 6\cr
{\sbf typ vysílání} & unicast & unicast & unicast & unicast & broadcast & broadcast\cr
}
\caption/t Přehled parametrů pro Cx scénáře.
\endinsert

Výsledky měření těchto scénáře si je možné prohlédnout v grafu \ref[ScenarioC-graph]. Zde se opět prokol chová dle očekávání. Nedochází k takovému zpoždění, jako například při měření scénářů typu A. Také si je možné povšimnou toho, že se jednou za čas nějaká zpráva opozdí.

Při tomto měření jsem zaznamenával také chybovost počet chybných zpráv.\fnote{Chybnou zprávou se myslí taková zpráva, která nedorazí do specifikovaného deadlinu, tedy mimo graf.} Jejich četnost si ji možné prohlédnout v tabulce \ref[ScenariosC-tab].

Během tohoto měření jsem zjistil, že je důležité, aby na větší vzdálenosti,\fnote{25 m a více} nestála signálu v cestě žádná překážka.

Zajímavé je také srovnání scénáře A a C. Můžeme pozorovat, že {\sbf vliv vzdálenosti ovlivňuje především ztrátovost paketů. Naopak velikost ovlivňuje rychlost přenosu.}

\secc Scénář D

\medskip
\clabel[ScenarioD-img]{Vizualizace parametrů pro měření scénářů D}
\picw=14cm \cinspic img/ScenarioD.png
\caption/f Vizualizace parametrů pro měření scénářů D.
\medskip

Scénář D byl oproti ostatním měřením odlišný v tom, že jsem se nejprve snažili stanovit hranici, kdy je zařízení ještě schopno přijímat zprávy a kdy už ne. Experimentálně jsem dospěl k hranici 580 m. Následně jsem odeslal 1000 zpráv s cílem zjistit, jak je veliká ztrátovost. Měření bylo realizováno na poli, přes které může procházet signál na 2,4 GHz.

\midinsert \clabel[ScenariosD-tab]{Přehled parametrů pro D scénář}
\ctable{lr}{
\hfil {\sbf SCÉNÁŘE} & {\sbf D1}\crl
{\sbf prostředí} & pole (s 2,4 GHz)\cr
{\sbf překážka} & vzduch\cr
{\sbf vzdálenost} & 577 m\cr
{\sbf velikost} & 125 B\cr
{\sbf počet zpráv} & 1 000\cr
{\sbf počet chybných zpráv} & 50\cr
{\sbf typ vysílání} & unicast\cr
}
\caption/t Přehled parametrů pro D scénář.
\endinsert

Při měření jsme zjistil, že při odesílání na velikou vzdálenost je třeba dbát na orientaci čipu. Pokud nebylo zařízení správně natočeno, nešlo odeslat žádné zprávy.

Výsledek měření je vizualizován grafem \ref[ScenarioD-graph]. Při tomto scénáři bylo ovšem mnohem zajímavjěíš pozorovat četnost úplné ztráty dat.

Při odeslání 1000 zpráv, se ztratilo 50. Můžeme tedy jednoduchým výpočtem zjistit jaká je procentuální ztrátovost na dlouhé vzdálenosti.

 $$
 loss = error / sent
 $$

V našem případě se při odeslání 1000 zpráv objevilo 50 chyb. Chybovost je tedy 5~\%.\fnote{$50/1000 = 0,05$}.

\sec Hodnocení měření

Veškerá naměřená data se skripty pro vyrenderováni grafů je možné najít v repositáři projektu.\fnote{\url{https://github.com/petrkucerak/rafting-button/tree/main/measure}}

Uvědomuji si, že měření nebylo prováděno za ideálních podmínek. Přesnost měření mohlo ovlivnit především to, že:

\begitems
* se jeden čipů měl větší odběr proudu a mírně se při měření přehříval,
* větší vzdálenosti\fnote{více jak 10 m} byly měřeny s přesností na jednotky metrů a největší vzdálenosti\fnote{více jak 100 m} s přesností na 10 m,
* les nebyl absolutně odstíněný od 2,4 GHz,
* během měření různých scénářů jsme upravoval kód,
* v některých scénářích by větší množství dat, mohlo říci více.

Zajímavé jsou ještě 2 výše zmíněné poznání. Konkrétně to, že při přenosu na delší vzdálenost je nutné, aby signálu nebránilo nic v cestě a aby byly moduly správně orientovány.

Výsledky prokazují, že protokol ESP-NOW splňuje očekávané požadavky a je vhodný pro použití v budoucí implementaci.