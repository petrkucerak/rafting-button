% Lokální makra patří do hlavního souboru, ne sem.
% Tady je mám výjimečně proto, že chci nechat hlavní soubor bez maker,
% která jsou jen pro tento dokument. Uživatelé si pravděpodobně budou
% hlavní soubor kopírovat do svého dokumentu.

\def\ctustyle{{\ssr CTUstyle}}
\def\ttb{\tt\char`\\} % pro tisk kontrolních sekvencí v tabulkách

\label[SpecifikacePozadavku]
\chap Specifikace funkčních a dalších požadavků na~systém

Kapitola navazuje na popis finálního řešení v~kapitole \ref[ButtonConceptDescription], specifikuje další požadavky a zdůvodňuje jednotlivé stanové parametry.

Ne všechny funkční požadavky jsou v této bakalářské práci realizované. Pro kompletní pochopení zamýšleného konceptu se domnívám, že je nutné je zde specifikovat.

\label[functionsRequirements]
\sec Funkční požadavky na systém

Výsledný systém by měl sloužit k {\sbf určení pořadí} toho, jak se jednotlivé týmy přihlásily o slovo, respektive seřazení událostí stisku tlačítek na jednotlivých zařízení v systému. Zařízení by měla být kompaktní ale zároveň odolná, a to i proti silným nárazům. Měla by fungovat minimálně 6 hodin v kuse.

Systém bude tvořen maximálně deseti zařízeními, a to z důvodu toho, že při podobných typech her nebývá více než 6 týmů. Měl by tedy limit deseti vyhovovat.

Zařízení by měla fungovat ve dvou módech: {\em PRESENTER} a {\em NORMAL}. V módu {\em NORMAL} funguje normálně, tedy řeší určení pořadí. V módu {\em PRESENTER} navíc vykresluje výsledky, a to pomocí webového serveru, na které se může uživatel připojit pomocí otevřeného {\em wifi access pointu}.

Stav hlasování bude signalizován LED diodami, přičemž může být ve dvou stavech {\em ACTIVE} a {\em USED}. První možnost znamená, že v daném kole se tým ještě nepřihlásil o slovo. Druhá, že se již o slovo přihlásil.

\midinsert \clabel[statuspossibilities]{ Možné stavy zařízení}
\ctable{cc}{
\hfil {\sbf mód zařízení} & {\sbf stav hlasování} \crl
{\em PRESENTER} & {\em ACTIVE}\cr
{\em NORMAL} & {\em USED}\cr
}
\caption/t Možné stavy zařízení.
\endinsert

Zařízení se ovládají pomocí hlasovacího tlačítka, které by mělo být odolné i vůči silným nárazům.\fnote{Koncept odolné krabičky pro celou skupinu je dle výsledků drobné ankety popsané v kapitole \ref[Obal] více žádaný než hlasovací tlačítko, které budou uživatelé moci držet v ruce.}

Zařízení v systému budou komunikovat pomocí drobných zpráv.\fnote{V řádech desítek Bajtů.}

Požadavky na přesnost rozlišení jsou takové, aby bylo možné zaručit správnost pořadí.\fnote{Daná problematika přesnosti je dále diskutovaná v kapitole \ref[syncspecdetailstime], která se podrobně stanovuje požadavky na algoritmus, konkrétně na synchronizaci času.}

Celý systém by měl fungovat v prostředí hospody, kde může být rušen cizím signálem. Musí také fungovat bez jakékoliv jiné infrastruktury například v lese.

Bezdrátová komunikace musí fungovat pomocí wifi technologie,\fnote{Wifi technologií se myslí {\em IEEE 802.11}.} která bude užívaná na co nejnižší úrovni.\fnote{Ideálně adresovat pomocí MAC adres a nevyužívat nic vyššího nad druhou spojovou vrstvu OSI modelu.}

\sec Popis užití systému v praxi

Workflow hry se systémem by měl být následující.

Moderátor hry položí otázku. Zařízení jsou na stolech před jednotlivými týmy. Týmy~se poradí. Jakmile~se domnívají, že znají správnou odpověď, stisknou tlačítko a~přihlásí se tak o slovo.

V ten moment se zaznamená daná událost, algoritmus ji zpracuje a v rámci celého distribuovaného systému určí její pořadí v daném kole. Výsledky jsou průběžně zobrazované na zařízení v módu {\em PRESENTER}.

Pro ukončení kola se na jakémkoliv zařízení stiskne tlačítko po delší dobu.\fnote{Například 10 s. Tato hodnota musí být později doladěna podle uživatelského zážitku, což není obsahem této bakalářské práce.} Tím se na všech zařízeních vymaže historie a systém je připrav pro další kolo.
