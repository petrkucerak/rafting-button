% Lokální makra patří do hlavního souboru, ne sem.
% Tady je mám výjimečně proto, že chci nechat hlavní soubor bez maker,
% která jsou jen pro tento dokument. Uživatelé si pravděpodobně budou
% hlavní soubor kopírovat do svého dokumentu.

\def\ctustyle{{\ssr CTUstyle}}
\def\ttb{\tt\char`\\} % pro tisk kontrolních sekvencí v tabulkách

\chap Úvod

V první kapitole bych chtěl vysvětlit motivace, které mne vedli k volbě tohoto téma. Rád bych popsal téma bakalářské práce, k níž tato práce směřuje. V poslední části úvodní kapitoly popíšu plány, jak bych rád v práci postupoval a jakých výsledků dosáhl.

\sec Motivace

Už od útlých dětských let jezdím na letní tábory a přes rok organizuji různé mládežnické akce. Na jedné z akcí, Hudebním týdnu, vždy hráváme hru, v které jsou různé týmy, jenž mezi sebou soutěží. Mají za úkol uhádnout nějakou otázku nebo např. poznat film dle hudby. Nikdy jsme nevymysleli efektivní způsob, jak poznat, kdo se o slovo přihlásil jako první. Vždy to skončí na zvonečku. V situaci, kdy se současně o slovo přihlásí více týmů, nejsme schopni určit pořadí. 

\sec Popis cíle bakalářské práce

Rád bych vytvořil hlasovací zařízení (tlačítko), které by fungovalo jako autonomní systém, který by společně s dalšími tlačítky tvořil distribuovanou síť. Síť by následně řešila problematiku konsenzu. Respektive pomocí vhodně navrženého a implementovaného algoritmu by bylo možné s určitostí tvrdit, jaké z tlačítek bylo stištěno první. 

Představa uživatelského zážitku je taková, že uživatel zapne hlasovací zařízení, zařízení se spojí s dalšími zařízeními v distribuované síti, uživatel má na tlačítku možnost aktivovat lokální server pro zobrazení pořadí stisknutí. 

\sec Samostatný projekt

Před navrhnutím samotného algoritmu pro řešení konsenzu je ale třeba specifikovat klíčové parametry, které bude muset síť zvládat. Proto jsem se v tomto samostatném projektu rozhodl vybudovat síťovou infrastrukturu pro následnou aplikaci při samotné bakalářské práci. 


\sec Stanovení cílů a plánů

Jelikož se jedná o komplexní problém, který čítá mnoho oblastí problematik, rozhodli jsme se, po diskuzi s vedoucím práce, práci rozdělit na několik fází.


\begitems
* Proveďte rešerši využití Wifi technologie pro IoT aplikace.
* Definujte kritické vlastností Wifi sítě (spolehlivost, odezva, propustnost sítě, …).
* Zvolte vhodný Wifi modul pro implementaci.
* Experimentálně ověřte dosažení výše definovaných kritických parametrů.
\enditems
