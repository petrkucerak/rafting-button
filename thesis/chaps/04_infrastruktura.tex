% Lokální makra patří do hlavního souboru, ne sem.
% Tady je mám výjimečně proto, že chci nechat hlavní soubor bez maker,
% která jsou jen pro tento dokument. Uživatelé si pravděpodobně budou
% hlavní soubor kopírovat do svého dokumentu.

\def\ctustyle{{\ssr CTUstyle}}
\def\ttb{\tt\char`\\} % pro tisk kontrolních sekvencí v tabulkách

\chap Síťová infrastruktura

Kapitola popisuje volbu vhodného modulu, návrh, realizaci a měření parametrů síťové infrastruktury.\fnote{Kapitola částečně vychází z práce a závěrů mého závěrečném projektu absolvovaném v rámci zimního semestru 2022 na ČVUT FEL.}

\sec Volba vhodného modulu

Požadavky na síťovou infrastrukturu jsou takové, že by měla zařízení komunikovat pomocí wifi technologie a drobných zpráv. Těmto parametrům je nutné podřídit volbu vhodného hardwaru. Pro něj navíc platí, že potřeba, aby zařízení mohlo fungovat minimálně po dobu 6 hodin pouze na baterie. Nedává smysl tedy přemýšlet o zařízeních s vyšším výkonem a větší spotřebou ale spíše se zaměřit na drobné moduly a mikrokontroléry splňující definované požadavky.

Zařízení by tedy mělo:

\begitems
* podporovat standardy IEEE 802.11 (b/g/n),
* mít možnost pracovat s MAC frames,
* mít pin pro zapojení tlačítek a dalších nutných periferií.

\enditems

Z průzkumu trhu mi vyšlo šest možných zařízení,\fnote{ESP32, ESP32-S2, ESP8266, PI Pico, {\em CC3200} a {\em ATSAMW25}.}. Velice rychle jsem zavrhl poslední dvě zmiňované, a to především z důvodu toho, že předchozí čtyři zařízení se těší větší oblibě a na internetu je větší komunita, která mi v případě nejasností s modulem může poradit.

\midinsert \clabel[possibleDevices]{Srovnání rozdílných parametrů modulů}
\ctable{lccc}{
\hfil {\sbf název} & {\sbf CPU} & {\sbf Flash} & {\sbf RAM} \crl
{\sbf ESP32}& dual-core Xtensa LX6 & pouze externí & 320 kB\cr
Espressif  & (160 nebo 240 MHz) &  & (SRAM)\crl
{\sbf ESP32-S2}  & single-core Xtensa LX7 & pouze externí & 320 kB\cr
Espressif  & (240 MHz) & & (SRAM)\crl
{\sbf ESP8266} & Xtensa L106 & pouze externí & 32 kB instrukce\cr
Espressif & (80 nebo 160 MHz) & & 80 kB data\crl
{\sbf Pico W} & dual-core Cortex-M0+ & 16MB & 264kB\cr
Raspberry Pi & (133 MHz) &(off-chip flash)\cr
}
\caption/t Srovnání rozdílných parametrů modulů.
\endinsert

Všechna zařazení kromě informací v tabulce \ref[possibleDevices] podporují požadovanou verzi IEEE 802.11 b/g/n.

Pro testování síťové infrastruktury jsem se rozhodl pro zařízení ESP32-S2, a to především z důvodu rychlosti procesoru, optimalizaci spotřeby energie a dostupnosti na trhu. Konkrétně se jednalo o moduly {\em ESP32-S2-pico} a {\em ESP32-S2-LCD-0.96inch} s displejem pro účely debugování.

Přesněji se jedná o wifi vývojové desky s základními periferiemi jako je ADC převodník, I2C a SPI komunikace či UART. Deska integruje {\em low-power Wifi} {\em System on Chip} (SoC). Oproti od ESP32, které má 2 jádra procesoru, ESP32-S2 má pouze jeden Xtensa single-core 32-bit procesor novější generace a podporuje frekvenci hodin 240 MHz.

Další specifikace je možné najít na stránkách dodavatele.\fnote{\url{https://www.waveshare.com/wiki/ESP32-S2-Pico}} Do přílohy \ref[SchemaModulu] přikládám relevantní schémata.

V pozdější fázi projektu při implementaci algoritmu jsem byl nucen přejít na ESP32,\fnote{Konkrétně se jednalo o vývojové desky ESP-WROOM-32 ESP32 ESP-32S 2.4GHz dostupnou na \url{https://www.briv.cz/p/5349/esp-wroom-32-esp32-esp-32s-2-4ghz-vyvojarska-deska-s-wifi-bt} a ESP-WROOM-32 2.4GHz Dual-Mode WiFi+Bluetooth rev.1, CP2102 dostupnou na \url{https://www.laskakit.cz/iot-esp-32s-2-4ghz-dual-mode-wifi-bluetooth-rev-1--cp2102/}.} a to z důvodu poškození modulů ESP32-S2. Díky této změně jsem ze single-coru přešel na dual-core. Umožnilo mi to lépe rozložit výpočetní kapacitu a zefektivnit tak celý provoz zařízení. Negativem byla implementační náročnost, protože přibyla nutnost užití RTOS, konkrétně FreeRTOS.

\sec Požadavky na síťovou infrastrukturu

Pro správné fungování zařízení, je třeba vybudovat síťovou infrastrukturu a zjistit její klíčové parametry. To především z důvodu abstrakce problému pro snadnější řešení dalších úkolů jako je například algoritmické řešení.

Síť by dle požadavků v kapitole \ref[functionsRequirements] měla komunikovat pomocí wifi technologie, která bude fungovat na co nejnižší úrovní. Informace by se měli posílat pomocí drobných zpráv. Síť musí fungovat i v prostředí bez jakékoliv jiné infrastruktury. Nedává tudíž smysl využívat typickou hierarchickou strukturu sítí.\fnote{Více informací o hierarchické struktuře sítě je možné dohledat v kapitole \ref[hieearchyStruct], která se dané problematice přímo věnuje.}

\sec Protokol ESP-NOW

Při studiu možností IEEE 802.11 pro IoT zařízení, jsem se setkal pouze s hierarchickou síťovou strukturou. Proto jsem se rozhodl pro distribuovanou síť, kde si budou všechna zapojená zařízení na počátku rovna. Tomuto rozhodnutí také požadavek, aby technologie fungovala nezávisle na jiné síťové infrastruktuře. To by v případě hierarchické struktury bylo třeba.

Při tvorbě rešerše jsem objevil nemalé množství protokolů, s nimiž by mohla zařízení mezi sebou komunikovat. Nejvhodnější pro mou aplikaci byly MQTT protokol či klasický TCP/IP. Po diskusi s vedoucím práce jsem se rozhodl pro experimentálnější způsob. Nepoužiji žádný ze zmiňovaných protokolů, nýbrž se pokusím implementovat vlastní protokol, který by dle definice IEEE 802.11\fnote{Konkrétně 2 úrovně síťového modelu OSI (linkové).} posílal MAC frames. Celá síť by byla adresovaná na úrovní MAC adres.\fnote{Nepovedlo se mi totiž žádný takový existující protokol vyhledat. Ani po diskusi s odborníky na telekomunikační technologie z řad učitelů, jsem se nedozvěděl o jakémkoliv již funkčním způsobu posílání dat pouze na 2 úrovni síťového OSI modelu.}

Po podrobnější analýze jsem zjistil, že implementovat vlastní protokol je zcela mimo rozsah této bakalářské práce. Objevil jsem ale protokol ESP-NOW, který je definovaný přímo pro mnou zvolené moduly, zajišťuje komunikaci pomocí malých zpráv s nízkou latencí, funguje na nízké vrstvě OSI modelu.

\secc Popis technologie

Protokol {\sbf ESP-NOW} je definovaný firmou Espressif speciálně pro procesory ESP32, ESP8266, ESP32-S a ESP32-C. Technologie se od Wifi či Bluetoothe LE liší tím, že horních pět vrstev OSI modelu spojuje do jedné zvané ESP-NOW. Díky tomu je protokol sice méně robustní, naopak je díky němu možné dosáhnout vyšších rychlostí za menší cenu výpočetní síly.

\medskip
\clabel[ESPNOW-ProtocolStack]{Vizualizace zasazení protokolu ESP-NOW do OSI modelu}
\picw=16cm \cinspic img/protocol-stack.png
\caption/f Vizualizace zasazení protokolu ESP-NOW do OSI modelu.~\cite[E7xNXczajtMU8qxH]
\medskip

Protokol je flexibilní, protože dokáže přenášet data jak v {\em unicast} tak v {\em broadcast} módu a podporuje jak {\em one-to-many}, tak {\em many-to-many} komunikaci.

\secc Formát rámce

Výchozí přenosová rychlost ({\em bit rate}) ESP-NOW je 1 MBps. Rychlost lze měnit pomocí funkcí definovaných protokolem. Formát rámce vychází z definice {\em IEEE 802.11},\fnote{{\em IEEE 802.11} kapitola {\em 9.6.7.11 Vendor Specific Public Action frame format}.} konkretně tzv. {\em Action Format}.~\cite[2zPvvpoRaPgJFnf1, 9363693]

\begitems
* {\sbf MAC Header} (24 bytes)
* {\sbf Category Code} (1 byte) indikující typ {\em Action Frame}. V případě ESP-NOW se jedná o tzv. {\em vendor-specific action frame}, a proto je hodnota nastavena na 127.
* {\sbf Organization Identifier} (3 bytes) je jedinečný identifikátor. V případě ESP-NOW jde o hodnotu "0x18fe34", která tvoří první 3 byty MAC adresy příslušící firmě Espressif.
* {\sbf Random Values} (4 bytes) je náhodná hodnota, která se používá k prevenci před relay útokem.\fnote{{\sbf Relay útok} je kyberbezpečnostní útok využívající techniky {\em men-in-the-middle}. Pro více podrobností doporučuji navštívit \url{https://en.wikipedia.org/wiki/Relay_attack}.}
* {\sbf Vendor Specific Content} (7{\sim}257 bytes) samotný obsah protokolu, který je tvořen vloženým framem.
    \begitems
    * {\sbf Element ID} (1 byte) hodnota je nastavena na "221" pro indikaci začátku {\em vendor-specific} části.
    * {\sbf Length} (1 byte) je celková velikost částí {\em Organization Identifier}, {\em Type}, {\em Version} a {\em Body}.
    * {\sbf Organization Identifier} (3 bytes) stejná hodnota jako v nadřazeném rámci, do kterého je tento zapouzdřen. Jedná o hodnotu "0x18fe34", která odpovídá prvním třem bajtům MAC adres příslušících firmě Espressif.
    * {\sbf Type} (1 byte) určuje typ action rámce od Espressifu. V případě užití ESP-NOW je hodnota rovna "4".
    * {\sbf Version} (1 byte) pole pro nastavení konkrétní verze.
    * {\sbf Body} (0{\sim}250 bytes\fnote{V oficiální dokumentaci byla v době kdy jsem na práci pracoval chyba. Body bylo definované tak, že může dosahovat velikosti až 250 bajtů. To ale není možné, jelikož součet by přesáhl maximální povolenou velikost, tedy 255 bajtů pro {\em Vendor Specific Content}. Chybu jsem nahlásil a věřím, že bude co nejdříve opravena. \url{https://github.com/espressif/esp-now/issues/59}}) tělo samotného rámce obsahující zprávu.
    \enditems
* {\sbf FCS} (4 bytes) frame check sequence
\enditems

\medskip
\clabel[ESP-NOW-Frame-01-img]{ESP-NOW Action Frame}
\picw=14cm \cinspic img/ESP-NOW-frame-01.png
\caption/f Vizualizace ESP-NOW Action Frame.
\medskip

\medskip
\clabel[ESP-NOW-Frame-02-img]{ESP-NOW Content Frame}
\picw=14cm \cinspic img/ESP-NOW-frame-02.png
\caption/f Vizualizace ESP-NOW Content Frame.
\medskip

Protokol ESP-NOW má několik dalších odlišností oproti klasické komunikaci dle IEEE 802.11. Liší se kontrolním rámcem ({\em Frame Control})\fnote{Definovaném v {\em IEEE 802.11} kapitola {\em 9.2.4.1 Frame Control field}} v tom, že bity "FromDS" a "ToDS" jsou nulové a v obecném rámci ({\em General frame formát})\fnote{Definovaný v {\em IEEE 802.11} kapitola {\em 9.2.3 General frame format}}, pro který platí, že:

\begitems
* v první adrese ({\em Address 1}) je uložena cílová destinace ve formě MAC adresy,
* druhá adresa ({\em Address 2}) ukládá adresu zdroje
* a třetí adresa ({\em Address 3}) je vždy natavena na vysílání broadcastem, tedy na adresu "0xff:0xff:0xff:0xff:0xff:0xff".
\enditems

Protokol ESP-NOW je zabezpečen pomocí ECDH a AES128-CCM.\fnote{Definovaných v IEEE 802.11-2012.} Více podrobností o bezpečnosti je možné dohledat v dokumentaci \fnote{\url{https://docs.espressif.com/projects/esp-idf/en/latest/esp32/api-reference/network/esp_now.html\#security}} nebo přímo v oficiální repozitáři implementace protokolu.\fnote{\url{https://github.com/espressif/esp-now/tree/master}} Jelikož šifrování nebude při implementaci použito, nebudu se mu v této práce věnovat více.

\secc Módy provozu wifi

Moduly podporující ESP-NOW\fnote{ESP32 a ESP8266} umožňují dva módů provozu wifi technologie: Station mode a Access Point mode (Soft-AP mode).\fnote{Ve skutečnosti existuje více módů provozu wifi na ESP modulech. Jedná se ale o hybridy již zmíněných módů, proto se jim nebudu více do detaily věnovat.}

{\sbf Station mode} (STA) se na používá v případě, kdy se modul připojuje k access pointu jako je například router. Typickou aplikací je senzor, který například snímá teplotu v místnosti, je napojen na router, který poskytuje připojení k WLAN a v pravidelných intervalech odesílá data do clodu.

Opakem je {\sbf Access Point mode} (Soft-AP mode někdy také označován pouze zkratkou AP mode), který funguje jako router v síti a může k němu být připojeno více zařízení. Typickou aplikací může být například spuštění wifi a webového serveru na modulu. V momentě, kdy se do sítě připojí nějaké zařízení, webový server vykreslí specifikovanou stránku.\fnote{Poměrně pěkně a detailně je implementace Wifi technologie na ESP zařízeních popsána v hlavičkovém souboru pro framework {\em ESP-IDF}, který je dostupný na adrese \url{https://raw.githubusercontent.com/espressif/esp-idf/4f0769d2ed074ef770c6432565d6e5610124e9ea/components/esp_wifi/include/esp_wifi.h}}
Tento mód je v této práci vhodný pro využití implementaci stavu zařízení v módu {\em PRESENTER}.

Protokol ESP-NOW může běžet ve všech módech.

\secc Popis fungování

Před využíváním ESP-NOW je třeba aktivovat wifi a {\sbf inicializovat} nutné struktury pomocí funkce $"esp_now_init()"$. Dokumentace doporučuje striktně dodržet zmiňované pořadí, tedy prvně aktivovat wifi a až poté protokol ESP-NOW.

Pro odeslání dat je třeba mít informace o cílových zařízeních přidaná v tzv. $"esp_now_peer_info"$. Do této struktury je data možné přidávat pomocí funkce $"esp_now_add_peer()"$. Struktura uchovává MAC adresu zařízení, informace o zabezpečení komunikace a další parametry komunikace. Podle verze protokolu je struktura limitovaná počtem šifrovaných a nešifrovaných zařízení.\fnote{Přesné limity doporučuji sledovat přímo v oficiálním repozitáři protokolu ESP-NOW: \url{https://github.com/espressif/esp-now/tree/master}.} Ve verzi 2.1.0 se jedná o maximálně 20 nešifrovaných případně 17 šifrovaných zařízení. Velikost jde případně uživatelsky v kódu změnit. To ale může mít negativní vliv na rychlost fungování protokolu.

\medskip
\clabel[ESPNOWDataTransferingImg]{Schéma odesílání dat pomocí ESP-NOW}
\picw=15cm \cinspic img/ESP-NOW-data-transfering.png
\caption/f Schéma odesílání dat pomocí ESP-NOW.
\medskip

Samotné {\sbf odesílání a přijímání dat} je ilustrované na obrázku \ref[ESPNOWDataTransferingImg] a funguje pomocí tzv. {\em call-back} funkcí. Funkce jsou v podstatě {\em Wifi tasky} s vysokou prioritou, které je třeba uživatelsky definovat. Protokol nedefinuje žádnou strukturu pro ukládání a pokročilé zpracovávání dat. Odesílání pak funguje tak, že uživatel zavolá funkci $"esp_now_send()"$, která data odešle na druhou vrstvu protokolu nazývanou MAC layer. Protokol pomocí {\em call-back} funkce $"espnow_send_cb()"$ odešle status úspěšnosti operace. Ověření úspěšnosti odeslání se liší v závislosti na módu komunikace. V případě {\em broadcastu} je zaznamenané pouze úspěšné odeslání. V případě {\em unicastu} se čeká na potvrzení od adresáta.\fnote{Tato informace není nikde oficiálně dokumentována a byla zjištěna experimentálně.} Přijímáni dat funguje na podobném principu. Jakmile {\em Wifi task} zaregistruje příchozí data, zavolá {\em call-back} funkci $"espnow_recv_cb()"$, která data dle definice zpracuje.

Protokol podporuje další pokročilé funkce jako je například {\sbf změna rychlosti přenosu} či využívání {\sbf úsporného režimu vysílání} s metodou nastavení vysílacích intervalů.

\secc Limity technologie

Protokol ESP-NOW je definován poměrně nově. Proto dochází k neustálému vývoji, drobným změnám a posouvání limitů.

První verze (1.0.0) vychází v roce 2016 a dle manuálu\fnote{\url{https://www.espressif.com/sites/default/files/documentation/esp-now_user_guide_en.pdf}} je limitována tím, že:

\begitems
* nemá implementovaný broadcast,
* je možné šifrovat spojení pouze s 10 zařízeními
* a maximální payload zprávy je 250 bajtů.
\enditems

Další verze rozšiřují počet šifrovaných zařízení, rychlost přenosu a přidávají podporu pro broadcast. Tyto limity se dle mého názoru budou dále rozvíjet. Vývojáři mají v nejbližší době v plánu knihovnu rozšiřovat o možnost clusterizování sítě.

Osobně také vidím nemalý problém v nedostatečné dokumentaci a vysvětlení, jak celý protokol funguje. Uživatel je nucen detailně zkoumat kódy a následně chování testovat, protože některé části nejsou veřejně dostupné.

Limit, který se dle mého názoru nezmění, tak je velikost zprávy. Ta je dána typem využívaného IEEE 802.11 rámce.

\label[PseudoBroadcast]
\secc ESP-NOW pro Arduino framework

Oficiální implementaci protokolu je možné aplikovat pouze s využitím {\sbf IoT frameworku ESP-IDF}. Pro využití {\sbf Arduino frameworku} je dostupná neoficiální verze od autora {\em Junxiao Shi} v repozitáři na GitHubu.\fnote{\url{https://github.com/yoursunny/WifiEspNow}} Tato verze není oficiálně udržovaná a v době psaní této bakalářské práce má z mého pohledu dva zásadní limity.

\begitems
* Knihovna {\sbf vychází ze zastaralé verze} (ESP-NOW 1.0.0). Tudíž neobsahuje opravy a nové funkce, které přináší až druhá verze protokolu.
* Knihovna implementuje {\sbf vlastní pseudo {\em {\sbf broadcast}}}. Ten funguje tak, že provede {\em BSSID}\fnote{Zkratka BSSID znamená {\sbf Basic Service Set Identifier} neboli 48-bitový identifikátor, pod kterým je zařízení identifikované v síti.~\cite[esBbWrhMVbvJ2grS]} skenování a na každé nalezné zařízení odešle zprávu pomocí {\em unicastu}. 
\enditems

Arduino Framework se nicméně těší vysoké oblibě. Z toho vyplývá, že většina veřejně dostupných návodů na internetu jsou nevhodné zdroje, protože popisují špatnou verzi.

\sec Architektura síťové infrastruktury

Komunikace mezi zařízeními je postavena na protokolu ESP-NOW. Ten hravě splňuje omezení 10 zařízení, drobných zpráv, i komunikace pomocí wifi na nízké úrovní OSI modelu. Pro zpracování je nutné využít pomocné struktury. Konkrétně se jedná o frontu, která pomáhá zpracovávat příchozí zprávy. Celý proces putování dat 
 během komunikace je ilustrován na obrázku \ref[InfraDataWorkflow] a probíhá tak, že:

\begitems
* před odesláním zprávy se připraví data, zkontroluje se zdali je cílové zařízení přidáno mezi známými destinacemi a dojde k odeslání pomocí $"send_message()"$,
* odesílatel obdrží status, který zachytí {\em call-back} funkcí $"espnow_send_cb()"$ a odešle informace do fronty
* příjemce obdrží zprávu, kterou zachytí pomocí {\em call-back} funkce $"espnow_recv_cb()"$ a data odešle do fronty.
\enditems

Data z fronty jsou následně vytažena taskem $"espnow_handler_task"$, který má za úkol data zpracovat a zavolat následnou navazující akci. V podstatě funguje jako takový {\em ESP-NOW event hub}.\fnote{koncentrátor událostí}

\medskip
\clabel[InfraDataWorkflow]{Workflow dat v síťové infrastruktuře}
\picw=14cm \cinspic img/infrastructure-data-workflow.png
\caption/f Workflow dat v síťové infrastruktuře.
\medskip

Postup byl zvolen z důvodu efektivity fungování, především prioritizace tasků a jejich efektivního rozložení mezi více jader. Nejvyšší prioritu mají vždy {\em call-back} funkce, po nich následuje {\em handler task} a nejnižší vždy má funkce sloužící k odesílání zpráv.

Navíc struktura $"QueueHandle_t"$ využívaná pro frontu je bezpečná. Díky ní není nutné pro zpracování událostí využívat více kritických sekcí.


\sec Parametry síťové infrastruktury

Pro možnost abstrakce síťové infrastruktury v další práci je nutné změřit a otestovat parametry, kterých daná síťová infrastruktura dosahuje, především vlastnosti modulu a protokolu ESP-NOW. Testování bylo prováděno na zařízeních ESP32-S2 v různých prostředích a za různé konfigurace.


\sec Program pro testování round-time trip

V závěrečné části jsem navrhnu a implementoval program pro měření round-time tripu. Fungování je detailněji popsáno v kapitole \ref[UvodKMereni].

Kód pro měření včetně Python skriptu pro vizualizaci výsledků je dostupný ve~verzi~{\em 3.0.3}\fnote{\url{https://github.com/petrkucerak/rafting-button/tree/3.0.3}}.