% Lokální makra patří do hlavního souboru, ne sem.
% Tady je mám výjimečně proto, že chci nechat hlavní soubor bez maker,
% která jsou jen pro tento dokument. Uživatelé si pravděpodobně budou
% hlavní soubor kopírovat do svého dokumentu.

\def\ctustyle{{\ssr CTUstyle}}
\def\ttb{\tt\char`\\} % pro tisk kontrolních sekvencí v tabulkách

\label[Obal]
\chap Zařízení z pohledu UX

Ve své práci jsem se okrajově věnoval i návrhu zařízení z pohledu UX.\fnote{User Experience - uživatelského zážitku} Na základě pravidelných diskuzí s kolegy z univerzity, přáteli a vedoucím mojí práce vyplynuly v podstatě dva možné koncepty.

V prvním případě se jednalo o zařízení {\sbf menšího} rázu, které by si mohl uživatel vzít do ruky a ovládat ho stiskem tlačítka. Druhou možností bylo vyrobit {\sbf robustnější} krabičku, kterou by šlo položit na stůl. Kdokoliv z uživatelů by mohl hlasovat například i pomocí silného úderu.

Původně jsem chtěl připravit dva prototypy, každý pro jednu verzi. Pomocí nich bych následně provedl výzkum na základě uživatelského zážitku. Z časových důvodů a také důvodu toho, že se daná práce orientuje na jinou oblast, jsem se po doporučení vedoucího práce rozhodl připravit pouze krátký formulář s cílem zjistit mínění lidí.

Formulář jsem koncipoval jednoduše. Na začátku popisoval motivaci a dva koncepty, které jsem doplnil o ilustrační obrázky vytvořené modelem AI.\fnote{Použil jsem model {\em DALL-E} napojený na aplikaci {\em Microsoft Designer}.} Následně jsem položil otázku, jakou z možností by korespondenti volili. Odpověď bylo možné doplnit komentářem.

\medskip
\clabel[ObalImg]{Ilustrační obrázky obalu vygenerované modelem AI}
\picw=15cm \cinspic img/Bakalarka-designes.png
\caption/f Ilustrační obrázky obalu vygenerované modelem AI.
\medskip

Formulář jsem sdílel ve story na svém instagramovém profilu. Za 24 hodin mi na formulář odpovědělo 105 lidí, přičemž 44 doplnilo svoji volbu o slovní komentář. Jsem si vědom, že lidé, kteří odpovídali, jsou z mého okolí. Tato skupina je z většiny tvořena mladými lidmi ve věku od 16 do 30 let, kteří studují nebo čerstvě dokončili vysokou školu. Jsou mezi nimi jak lidé z technických oborů, tak ale i lidé z oblasti školství či zdravotnictví. Jelikož nepracuji s reprezentativním vzorkem, rozhodně nelze data považovat za zcela empirická. Domnívám se, že se ale mohu opřít o slovní komentáře, které jsou přiloženy v příloze \ref[UXFormAsnwares].

\medskip
\clabel[ObalVysledkyImg]{Graf výsledků z průzkumu podoby zařízení}
\picw=15cm \cinspic img/Bakalarka-designes-chart.png
\caption/f Graf výsledků z průzkumu podoby zařízení.
\medskip

Z výsledků můžeme vyvodit, že lidé spíše preferují jedno větší zařízení pro celý tým.