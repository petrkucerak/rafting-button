% Lokální makra patří do hlavního souboru, ne sem.
% Tady je mám výjimečně proto, že chci nechat hlavní soubor bez maker,
% která jsou jen pro tento dokument. Uživatelé si pravděpodobně budou
% hlavní soubor kopírovat do svého dokumentu.

\def\ctustyle{{\ssr CTUstyle}}
\def\ttb{\tt\char`\\} % pro tisk kontrolních sekvencí v tabulkách

\label[Algoritmus]
\chap Algoritmus řešící konsenzus

Kapitola specifikuje požadavky na algoritmus, popisuje známé algoritmy vhodné pro daný problém, které dává do kontextu a diskutuje jejich výhody a nevýhody a detailně popisuje zvolený algoritmus pro implementaci.

\sec Požadavky na algoritmus

Od algoritmu se očekává nejen řešení {\sbf problému konsenzu},\fnote{Pojmem problém konsenzu je myšlený problém nutnosti najít shody dvou až n procesů na specifické hodnotě. Ve vztahu k našemu tématu se jedná o shodě na logu určujícím pořadí.} respektive toho, kdo první stiskl tlačítko, ale i konsenzus při určení pořadí stisknutí tlačítek na jednotlivých zařízení.

Síť by měla být {\sbf distribuovaná}, tj. ve výchozím módu by měli být všechny zařízení na stejné úrovni a žádné z nich by nemělo vůči jinému být ve vztahu {\em master} - {\em slave}.

Pojmem {\sbf distribuovaná síť}, obecně {\sbf distribuovaný systém} (DS)\fnote{Pojem {\sbf distribuovaný systém} je dále označován i zkratkou {\sbf DS}.} označujeme soubor autonomních, nezávislých, zařízení, která spolu komunikují skrze síť a jejich dorozumívacím prostředkem jsou zprávy. Dle Andrewa Tanenbauma se v ideálním případě celý distribuovaný systém jeví jako koherentní systém. Leslie Lamport je naopak kritičtější a říká, že pojmeme distribuovaný systém je takový, v kterém selhání počítače, o kterém jste nevěděli, že funguje, učiní váš počítač nepoužitelným.~\cite[DocIngCyrilKlimesCSc2014, fiRjEXokpUGbjPGS, Jakob2021-08]

Literatura se v dané terminologii liší a také na každý z problémů v DS jsou kladeny jiné požadavky, obecně ale můžeme řici, že rozlišujeme dva základní požadavky na DS:

\begitems
* {\sbf živost} ({\em safety}) - časem v DS bude dosažen žádoucí vztah,
* {\sbf bezpečnost} ({\em liveness}) - nedojde k nežádoucímu stavu.
\enditems

Dle {\sbf FLP teorému} v asynchronním distribuovaném systému nelze dosáhnout současně živosti a bezpečnosti v distribuovaného výpočtu, pokud může docházet k selháním.~\cite[Fischer1985] V praxi se proto spíše vyžaduje bezpečnost a díky částečné synchronicitě zařízení v DS lze předpokládat, že algoritmus doběhne, tj. dosáhne výsledku v konečném čase.\fnote{Takovýto požadavek označujeme jako tzv. {\sbf {\em konečnou} živost} ({\em eventual liveness}).}~\cite[Jakob2021-13]

\secc Požadavky na distribuovaný systém

\begitems
* {\sbf Bezpečnost} - systém musí garantovat, že nedojde k nežádoucímu stavu, tj. stavu s kterým se nepočítá.
* {\sbf Konečnou živost} - není nutné, aby byl distribuovaný výpočet dokončen do určitého času. Nýbrž je požadováno dokončení výpočtu v konečném čase.
* {\sbf Uspořádání} - musí být dodrženo kauzální uspořádání v závislosti na čase, tj. musí být dodrženo pořadí.
\enditems

\sec Srovnání možných algoritmů

\secc Synchronizace fyzických hodin (PTP, NTP)

\secc Lamportovy vektorové hodiny



\sec Detailní popis zvoleného algoritmu