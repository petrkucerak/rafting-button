% Lokální makra patří do hlavního souboru, ne sem.
% Tady je mám výjimečně proto, že chci nechat hlavní soubor bez maker,
% která jsou jen pro tento dokument. Uživatelé si pravděpodobně budou
% hlavní soubor kopírovat do svého dokumentu.

\def\ctustyle{{\ssr CTUstyle}}
\def\ttb{\tt\char`\\} % pro tisk kontrolních sekvencí v tabulkách

\label[ESPNOWChap]
\chap Technologie ESP-NOW

Kapitola se věnuje popisu technologie ESP-NOW, na které je kompletně vybudovaná síťová infrastruktura celého řešení.

\sec Popis technologie

Protokol {\sbf ESP-NOW} je definovaný firmou Espressif speciálně pro procesory ESP32, ESP8266, ESP32-S a ESP32-C. Technologie se od Wifi či Bluetoothe LE liší tím, že horních pět vrstev OSI modelu spojuje do jedné zvané ESP-NOW. Díky tomu je protokol sice méně robustní, naopak je díky němu možné dosáhnout vyšších rychlostí za menší cenu výpočetní síly.

\medskip
\clabel[ESPNOW-ProtocolStack]{Vizualizace zasazení protokolu ESP-NOW do OSI modelu}
\picw=16cm \cinspic img/protocol-stack.png
\caption/f Vizualizace zasazení protokolu ESP-NOW do OSI modelu.~\cite[E7xNXczajtMU8qxH]
\medskip

Protokol je flexibilní, protože dokáže přenášet data jak v {\em unicast} tak v {\em brodcast} módu a podporuje jak {\em one-to-many}, tak {\em many-to-many} komunikaci.

\secc Formát rámce

Výchozí přenosová rychlost ({\em bit rate}) ESP-NOW je 1 MBps. Rychlost jde měnit pomocí metod definovaných protokolem.

Formát rámce vychází z definice {\em IEEE 802.11},\fnote{{\em IEEE 802.11} kapitola {\em 9.6.7.11 Vendor Specific Public Action frame format}.} konkretně tzv. {\em Action Format} a struktura je definována následovně:~\cite[9363693]

\begitems
* {\sbf MAC Header} (24 bytes)
* {\sbf Category Code} (1 byte) indikující typ {\em Action Frame}, v případě ESP-NOW se jedná o tzv. {\em vendor-specific action frame}, a proto je hodnota nastavena na 127
* Organization Identifier
* Random Values
* Vendor Specific Content
* FCS
\enditems
\cite[2zPvvpoRaPgJFnf1]

\secc Typy vysílání

\secc Bezpečnost a šifrování

\secc Limity technologie

\sec Navržení implementace