% Lokální makra patří do hlavního souboru, ne sem.
% Tady je mám výjimečně proto, že chci nechat hlavní soubor bez maker,
% která jsou jen pro tento dokument. Uživatelé si pravděpodobně budou
% hlavní soubor kopírovat do svého dokumentu.

\def\ctustyle{{\ssr CTUstyle}}
\def\ttb{\tt\char`\\} % pro tisk kontrolních sekvencí v tabulkách

\chap Úvod

V první kapitole vysvětluji motivace, které mne vedli k volbě daného téma. Specifikuji očekávané výstupy a výsledky, kterých bych chtěl dosáhnout a v závěru navrhuji samotný postup.

\sec Motivace

Už od útlých dětských let jezdím na letní tábory a přes rok organizuji mládežnické akce. Na jedné z akcí, Hudebním týdnu, vždy hráváme hru, v které jsou různé týmy, jenž mezi sebou soutěží. Cílem je se jako první přihlásit o slovo a zodpovědět otázku. Doposud se nám nepovedlo vymyslet efektivní způsob, jak určit pořadí týmů. Ve výsledku vždy použijeme zvoneček, který při odpovědí více týmů ve stejný čas není vhodným řešením.


\sec TODO: Popis cíle bakalářské práce

Rád bych vytvořil hlasovací zařízení (tlačítko), které by fungovalo jako autonomní systém, který by společně s dalšími tlačítky tvořil distribuovanou síť. Síť by následně řešila problematiku konsenzu. Respektive pomocí vhodně navrženého a implementovaného algoritmu by bylo možné s určitostí tvrdit, jaké z tlačítek bylo stištěno první. 

Představa uživatelského zážitku je taková, že uživatel zapne hlasovací zařízení, zařízení se spojí s dalšími zařízeními v distribuované síti, uživatel má na tlačítku možnost aktivovat lokální server pro zobrazení pořadí stisknutí.

\sec TODO: Samostatný projekt

Před navrhnutím samotného algoritmu pro řešení konsenzu je ale třeba specifikovat klíčové parametry, které bude muset síť zvládat. Proto jsem se v tomto samostatném projektu rozhodl vybudovat síťovou infrastrukturu pro následnou aplikaci při samotné bakalářské práci. 


\sec TODO: Stanovení cílů a plánů

Jelikož se jedná o komplexní problém, který čítá mnoho oblastí problematik, rozhodli jsme se, po diskuzi s vedoucím práce, práci rozdělit na několik fází.


\begitems
* Proveďte rešerši využití Wifi technologie pro IoT aplikace.
* Definujte kritické vlastností Wifi sítě (spolehlivost, odezva, propustnost sítě, …).
* Zvolte vhodný Wifi modul pro implementaci.
* Experimentálně ověřte dosažení výše definovaných kritických parametrů.
